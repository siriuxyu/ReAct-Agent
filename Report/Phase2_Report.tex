\documentclass[letterpaper]{article}

% AAAI style packages
\usepackage{aaai24}
\usepackage{times}
\usepackage{helvet}
\usepackage{courier}
\usepackage[hyphens]{url}
\usepackage{graphicx}
\urlstyle{rm}
\usepackage{natbib}
\usepackage{caption}
\frenchspacing
\setlength{\pdfpagewidth}{8.5in}
\setlength{\pdfpageheight}{11in}

% Additional packages
\usepackage{booktabs}
\usepackage{multirow}
\usepackage{amsmath}
\usepackage{listings}
\usepackage{xcolor}
\usepackage{tikz}

% Code listing style
\lstset{
    basicstyle=\ttfamily\small,
    breaklines=true,
    frame=single,
    backgroundcolor=\color{gray!10},
    columns=fullflexible
}

% PDF metadata
\pdfinfo{
/TemplateVersion (2024.1)
}

\setcounter{secnumdepth}{2}

\title{Context/Memory Implementation \& Performance:\\An Enhanced AI Agent with Long-Term Memory}

\author{
    Ziye Deng, Zheqi Liu, Zhengan Cheng, Tianzuo Liu, Hengjia Yu\\
    UC San Diego\\
}

\begin{document}

\maketitle

\begin{abstract}
We present an enhanced AI agent system with comprehensive context and memory capabilities. Our implementation features a hybrid extraction approach combining pattern-based and LLM-based methods, persistent vector storage using ChromaDB, and dynamic tool optimization based on user preferences. Experimental results demonstrate significant improvements over the baseline: 100\% task completion on all benchmarks (up from 21.4\% on long conversations), response quality improvements of up to 114.3\%, and 85.20\% accuracy on cross-session memory retrieval. The system provides effective personalization through automatic preference extraction and maintains coherence across extended multi-turn conversations.
\end{abstract}

\section{Implementation}

\subsection{Context Extraction from Conversations and Tool Outputs}

\textbf{Design Decision:} We employ a hybrid extraction approach combining pattern-based and LLM-based methods to balance real-time performance with extraction accuracy.

\paragraph{Pattern-Based Extraction}
Our pattern-based extraction system provides fast, rule-based extraction for common preference indicators. It enables real-time preference detection during conversations and classifies preferences into communication style, domain interests, response format, tool preferences, and interaction patterns. The system uses confidence scoring to prioritize high-quality extractions.

\paragraph{LLM-Based Extraction}
For deeper analysis, we leverage Claude's structured output capabilities for semantic analysis, capturing nuanced preferences that pattern matching might miss while providing structured preference data for storage and retrieval.

\textbf{Key Technical Challenge:} Balancing extraction speed (for real-time use) with accuracy (for quality personalization). The hybrid approach addresses this by using fast pattern matching for common cases and LLM extraction for complex scenarios.

\paragraph{Preference Management}
The preference management system includes:
\begin{itemize}
    \item Intelligent merging of preferences across sessions with weighted confidence scores
    \item Temporal tracking to identify preference evolution over time
    \item Deduplication to prevent redundant storage
\end{itemize}

\subsection{Storage System}

\textbf{Design Decision:} ChromaDB serves as the vector database backend for persistent, scalable semantic search.

\paragraph{Why ChromaDB}
We selected ChromaDB for several reasons: it is a lightweight, embedded database suitable for local deployment with native support for vector similarity search. It provides efficient metadata filtering for user isolation and requires no external service dependencies.

\paragraph{Key Technical Challenges Solved}

\begin{enumerate}
    \item \textbf{User Isolation:} ChromaDB's flat metadata structure required careful design to ensure complete user data isolation using \texttt{user\_id} metadata filters, preventing cross-user data leakage in multi-tenant scenarios.
    
    \item \textbf{Metadata Management:} Complex nested metadata (session IDs, document types, timestamps) needed flattening for ChromaDB compatibility while maintaining query flexibility.
    
    \item \textbf{Embedding Pipeline:} Integration with OpenAI embeddings (\texttt{text-embedding-3-small}) with batch processing to optimize API usage and reduce latency.
    
    \item \textbf{Semantic Search:} Cosine similarity search with configurable similarity thresholds to balance recall (finding relevant memories) with precision (avoiding irrelevant results).
\end{enumerate}

\paragraph{Storage Architecture}
The storage architecture follows a three-layer design: Agent Memory Tools (\texttt{search\_memory}, \texttt{store\_memory}) connect to the VectorStorageBackend (ChromaDB with persistent storage), which in turn interfaces with the EmbeddingService (OpenAI).

\subsection{Tool Call Optimization Based on Context}

\textbf{Design Decision:} Dynamic tool loading with user-specific memory tools and context-aware tool selection.

\textbf{Key Innovation:} Integration with LangMem enables per-user memory tool generation while maintaining complete namespace isolation. This allows the agent to have memory capabilities without exposing one user's data to another.

\paragraph{Context-Aware Optimization}
\begin{itemize}
    \item User preferences influence tool prioritization (e.g., language preferences trigger translator tool suggestions)
    \item Historical context retrieved through semantic search informs tool selection
    \item Memory tools automatically search user's past conversations for relevant information
\end{itemize}

\textbf{Technical Challenge:} Ensuring memory tools are dynamically created per-user while maintaining performance and isolation. LangMem's namespace-based isolation solves this by creating separate tool instances per user namespace.

\subsection{Error Handling and Resource Management}

\textbf{Design Philosophy:} Graceful degradation ensures the agent continues operating even if memory components fail, maintaining core functionality while logging errors for debugging.

\paragraph{Key Design Decisions}
\begin{enumerate}
    \item \textbf{Graceful Degradation:} If memory storage is unavailable, the agent falls back to stateless operation, ensuring reliability in production environments.
    
    \item \textbf{Rate Limit Management:} Configurable token limits and request delays prevent API rate limit violations, critical for production deployment with Anthropic's API constraints.
    
    \item \textbf{Resource Cleanup:} Proper connection management for ChromaDB ensures no resource leaks during long-running operations.
    
    \item \textbf{Structured Logging:} Comprehensive logging with request tracking, user context, and performance metrics enables debugging and performance monitoring.
\end{enumerate}

\section{Experimental Results and Performance Analysis}

\subsection{Benchmark Results}

We evaluated the enhanced agent on three benchmark datasets: \textbf{short}, \textbf{medium}, and \textbf{long} conversations, plus the \textbf{LoCoMo} cross-session memory benchmark.

\subsubsection{Evaluation Framework}

\paragraph{Benchmark Datasets}
\begin{itemize}
    \item \textbf{Short Benchmark} (6 cases): Basic tool usage validation, single-turn interactions, answer correctness validation
    \item \textbf{Medium Benchmark} (4 cases): Multi-turn conversations (2-4 turns), context retention testing, preference extraction validation
    \item \textbf{Long Benchmark} (4 cases): Extended conversations (9-11 turns), complex task completion, memory integration testing
    \item \textbf{LoCoMo Benchmark}: Cross-session memory evaluation, 19 sessions, 419 conversation turns, 196 QA questions across 5 categories
\end{itemize}

\paragraph{Evaluation Metrics}
\begin{itemize}
    \item \textbf{Accuracy Metrics:} Answer correctness (semantic matching), per-turn and per-case analysis
    \item \textbf{Performance Metrics:} Latency per turn, memory storage statistics, token usage and rate limiting
    \item \textbf{Quality Metrics:} Preference extraction success rate, context retention across turns, personalization effectiveness
\end{itemize}

\subsubsection{Short Benchmark Results}

Table~\ref{tab:short-results} presents the short benchmark results covering 6 test cases for basic tool usage.

\begin{table}[h]
\centering
\caption{Short Benchmark Results}
\label{tab:short-results}
\begin{tabular}{ll}
\toprule
Metric & Result \\
\midrule
Total Cases & 6 \\
Answer Correct & 6/6 (100\%) \\
Response Quality & 4.8/5.0 \\
Tool Call Efficiency & 1.00 \\
Average Latency & $\sim$14.0 seconds \\
\bottomrule
\end{tabular}
\end{table}

\paragraph{Key Observations}
The system achieved \textbf{perfect accuracy} on all test cases with \textbf{perfect tool call efficiency} (1.00). All expected tools were called with correct frequency. The agent successfully handled various task types including calculator, weather, translator, file search, and tool combinations, demonstrating context awareness in multi-turn conversations.

\subsubsection{Medium Benchmark Results}

Table~\ref{tab:medium-results} shows the medium benchmark results for 4 test cases with multi-turn conversations requiring context retention.

\begin{table}[h]
\centering
\caption{Medium Benchmark Results}
\label{tab:medium-results}
\begin{tabular}{ll}
\toprule
Metric & Result \\
\midrule
Total Cases & 4 \\
Answer Correct & 4/4 (100\%) \\
Response Quality & 4.775/5.0 \\
Tool Call Efficiency & 1.00 \\
Average Latency & $\sim$12.0 seconds \\
\bottomrule
\end{tabular}
\end{table}

\paragraph{Key Observations}
The system achieved \textbf{perfect accuracy} with excellent response quality (4.775/5.0) on multi-turn conversations. Successful context retention across 2-4 conversation turns was observed, along with effective preference extraction and personalization.

\paragraph{Key Improvements Demonstrated}
\begin{enumerate}
    \item \textbf{Perfect Context Retention:} Agent maintains conversation context across multiple turns (2-4 turns) with 100\% accuracy
    \item \textbf{Effective Preference Personalization:} User preferences extracted and applied in 2/4 cases, including \texttt{living\_place}, \texttt{hobbies}, and \texttt{preferred\_translation\_language}
    \item \textbf{High Response Quality:} 4.775/5.0 average quality rating demonstrates excellent response relevance and coherence
    \item \textbf{Multi-Turn Coherence:} Successfully handles complex multi-turn tasks including reading comprehension, trip planning, arithmetic, and translation
\end{enumerate}

\subsubsection{Long Benchmark Results}

Table~\ref{tab:long-results} presents the long benchmark results for 4 complex test cases with 9-11 conversation turns.

\begin{table}[h]
\centering
\caption{Long Benchmark Results}
\label{tab:long-results}
\begin{tabular}{ll}
\toprule
Metric & Result \\
\midrule
Total Cases & 4 \\
Answer Correct & 4/4 (100\%) \\
Response Quality & 4.65/5.0 \\
Tool Call Efficiency & 0.714 \\
\bottomrule
\end{tabular}
\end{table}

\paragraph{Key Observations}
The system achieved \textbf{perfect accuracy} on all test cases with \textbf{good tool call efficiency} (0.714). Most expected tools were called correctly, with some cases using alternative tools. Excellent response quality (4.65/5.0) was observed on extended conversations, with successful context maintenance across 9-11 conversation turns.

\paragraph{Key Improvements}
\begin{enumerate}
    \item \textbf{Long Conversation Handling:} Maintains coherence across 9-11 turns
    \item \textbf{Memory Integration:} Successfully stores and retrieves information using \texttt{store\_memory} tool
    \item \textbf{Preference Persistence:} Extracted preferences influence responses throughout long conversations
\end{enumerate}

\subsubsection{LoCoMo Cross-Session Memory Benchmark}

Table~\ref{tab:locomo-results} shows the cross-session memory evaluation results.

\begin{table}[h]
\centering
\caption{LoCoMo Benchmark Results}
\label{tab:locomo-results}
\begin{tabular}{ll}
\toprule
Metric & Result \\
\midrule
Personas Run & 1 \\
Total Chunks Stored & 144 \\
Total Characters Stored & 67,162 \\
QA Accuracy & 167/196 (85.20\%) \\
Response Quality & 4.29/5.0 \\
Verified Memories & 100 \\
\bottomrule
\end{tabular}
\end{table}

Table~\ref{tab:locomo-category} presents the category-wise breakdown of results.

\begin{table}[h]
\centering
\caption{LoCoMo Category Breakdown}
\label{tab:locomo-category}
\begin{tabular}{lccc}
\toprule
Category & Total & Correct & Accuracy \\
\midrule
Category 1 & 31 & 22 & 71.0\% \\
Category 2 & 37 & 33 & 89.2\% \\
Category 3 & 11 & 10 & 90.9\% \\
Category 4 & 70 & 62 & 88.6\% \\
Category 5 & 47 & 40 & 85.1\% \\
\bottomrule
\end{tabular}
\end{table}

\paragraph{Analysis}
The system achieved \textbf{high overall accuracy} (85.20\%) across all question categories with \textbf{good response quality} (4.29/5.0). Excellent performance was observed in specific categories (Categories 2-5: 85-91\% accuracy). Category 1 performance (71\% accuracy for identity/personal information questions) indicates that category-specific optimization for identity/personal information extraction would improve performance.

\textbf{Note on Temporal Questions:} The benchmark does not include exact timestamps of sessions when storing memories or querying the model. Therefore, the model is only expected to output \textbf{relative time} information (e.g., ``earlier'', ``in a previous session'') rather than absolute timestamps.

\subsection{Performance Improvements Over Baseline}

\subsubsection{Better Generation Quality}

Table~\ref{tab:task-completion} shows the task completion rate comparison between phases.

\begin{table}[h]
\centering
\caption{Task Completion Rate Comparison}
\label{tab:task-completion}
\begin{tabular}{lccc}
\toprule
Benchmark & Phase 1 & Phase 2 & Improvement \\
\midrule
Short & 94.3\% & 100\% & +5.7\% \\
Medium & 66.7\% & 100\% & +33.3\% \\
Long & 21.4\% & 100\% & +78.6\% \\
\bottomrule
\end{tabular}
\end{table}

Table~\ref{tab:response-quality} presents the response quality rating comparison.

\begin{table}[h]
\centering
\caption{Response Quality Ratings (5-point scale)}
\label{tab:response-quality}
\begin{tabular}{lccc}
\toprule
Benchmark & Phase 1 & Phase 2 & Improvement \\
\midrule
Short & 4.45/5.0 & 4.8/5.0 & +7.9\% \\
Medium & 3.85/5.0 & 4.775/5.0 & +24.0\% \\
Long & 2.17/5.0 & 4.65/5.0 & +114.3\% \\
\bottomrule
\end{tabular}
\end{table}

\paragraph{Key Observations}
\begin{itemize}
    \item \textbf{Short:} Small improvement from already high baseline (94.3\% $\rightarrow$ 100\%)
    \item \textbf{Medium:} Major improvement (+33.3\%), demonstrating better context retention in multi-turn conversations
    \item \textbf{Long:} Dramatic improvement (+78.6\%), showing the critical value of memory system for extended conversations
\end{itemize}

\subsubsection{Better Personalization Based on User Preferences}

\paragraph{Preference Extraction Success}
The system successfully extracted preferences in multiple test cases:
\begin{itemize}
    \item \texttt{living\_place}: San Diego
    \item \texttt{hobbies}: Diving
    \item \texttt{preferred\_translation\_language}: French, Spanish
    \item \texttt{teaching\_tone}: Casual/friendly
\end{itemize}

\paragraph{Personalization Examples}
\begin{enumerate}
    \item \textbf{Location-based personalization:} The agent responded ``Since I know you live in San Diego, let me compare the water temperatures for you...'' providing relevant comparisons between San Diego and Hawaii water temperatures.
    
    \item \textbf{Language preference:} Consistent translation language maintained across conversations with teaching style adapted to user's learning preferences.
\end{enumerate}

\subsubsection{Response Time/Latency}

\paragraph{Average Latency by Benchmark}
\begin{itemize}
    \item \textbf{Short:} $\sim$14.0 seconds per turn
    \item \textbf{Medium:} $\sim$12.0 seconds per turn
    \item \textbf{Long:} $\sim$17.0 seconds per turn (complex multi-tool operations)
\end{itemize}

\paragraph{Optimization Opportunities}
\begin{itemize}
    \item Batch embedding generation reduces per-request overhead
    \item Memory search is asynchronous and non-blocking
    \item Tool execution parallelization where possible
    \item Embedding caching to reduce repeated API calls
\end{itemize}

\subsection{Side-by-Side Comparisons with Baseline (Phase 1)}

\subsubsection{Phase 1 Baseline Characteristics}

\paragraph{Phase 1 Implementation}
\begin{itemize}
    \item \textbf{Architecture:} Basic ReAct pattern with LangGraph
    \item \textbf{Memory:} No long-term memory, only in-memory conversation history
    \item \textbf{Context:} Limited to current conversation window
    \item \textbf{Personalization:} No user preference extraction or storage
    \item \textbf{Tools:} Static tool set (calculator, weather, translator, web\_reader, file\_system\_search)
    \item \textbf{Session Management:} Basic thread-based isolation, no cross-session continuity
\end{itemize}

\paragraph{Key Limitations}
\begin{enumerate}
    \item No persistent storage---all context lost after session ends
    \item No user preference learning---generic responses for all users
    \item Limited context window---cannot maintain coherence in long conversations
    \item No cross-session memory---cannot recall information from previous sessions
\end{enumerate}

\subsubsection{Comprehensive Feature Comparison}

Table~\ref{tab:feature-comparison} presents a comprehensive feature comparison between phases.

\begin{table*}[t]
\centering
\caption{Feature Comparison Between Phase 1 and Phase 2}
\label{tab:feature-comparison}
\begin{tabular}{llll}
\toprule
Feature & Phase 1 (Baseline) & Phase 2 (Enhanced) & Improvement \\
\midrule
Long-term Memory & None & ChromaDB vector storage & New capability \\
Context Extraction & None & Pattern + LLM-based extraction & New capability \\
Preference Learning & None & Automatic extraction and storage & New capability \\
Cross-Session Recall & Not possible & 85.20\% accuracy (LoCoMo) & New capability \\
Personalization & Generic responses & User-specific adaptation & Significant improvement \\
Context Retention & Limited to session & Multi-turn + cross-session & Major improvement \\
Tool Optimization & Static tool set & Context-aware tool selection & Enhanced \\
Storage Backend & None & Persistent ChromaDB & New capability \\
\bottomrule
\end{tabular}
\end{table*}

\subsubsection{Performance Comparison}

\paragraph{Comparison 1: Context Retention}

\textbf{Phase 1 (Baseline):}
\begin{itemize}
    \item No long-term memory
    \item Limited context window (only current session)
    \item Cannot reference information from previous sessions
    \item Context lost when conversation ends
\end{itemize}

\textbf{Phase 2 (Enhanced):}
\begin{itemize}
    \item Maintains context across 9-11 turns (long benchmark)
    \item Persistent storage in ChromaDB
    \item Recalls user preferences from previous sessions
    \item References earlier conversation points across sessions
\end{itemize}

\textbf{Quantitative Improvement:}
\begin{itemize}
    \item Long conversations: Phase 1 struggled with coherence beyond 3-4 turns; Phase 2 maintains 100\% accuracy across 9-11 turns
    \item Cross-session: Phase 1 had 0\% capability; Phase 2 achieves 85.20\% accuracy on LoCoMo benchmark
\end{itemize}

\paragraph{Comparison 2: Personalization}

\textbf{Phase 1 (Baseline):}
\begin{itemize}
    \item Generic responses for all users
    \item No user-specific adaptation
    \item No preference learning
\end{itemize}

\textbf{Phase 2 (Enhanced):}
\begin{itemize}
    \item Extracts and stores user preferences
    \item Adapts responses based on stored preferences
    \item Maintains consistent personalization across sessions
    \item Context-aware tool selection
\end{itemize}

\paragraph{Comparison 3: Cross-Session Memory}

\textbf{Phase 1 (Baseline):}
\begin{itemize}
    \item No cross-session memory capability (0\% accuracy)
    \item Each session starts from scratch
\end{itemize}

\textbf{Phase 2 (Enhanced):}
\begin{itemize}
    \item Stores conversation summaries in ChromaDB
    \item Retrieves relevant memories using semantic search
    \item 85.20\% overall accuracy on LoCoMo benchmark
    \item 85-91\% accuracy in specific question categories
\end{itemize}

\subsubsection{Architecture Comparison}

\textbf{Phase 1 Architecture:}
\begin{center}
User Request $\rightarrow$ Agent (ReAct) $\rightarrow$ Tools $\rightarrow$ Response
\end{center}

\textbf{Phase 2 Architecture:}
\begin{center}
User Request $\rightarrow$ Agent (ReAct) $\rightarrow$ Context Extraction $\rightarrow$ Preference Storage\\
$\downarrow$\\
Tools $\leftarrow$ Memory Search $\leftarrow$ ChromaDB $\leftarrow$ Embeddings\\
$\downarrow$\\
Response (Personalized)
\end{center}

\paragraph{Key Architectural Improvements}
\begin{enumerate}
    \item \textbf{Memory Layer:} Added persistent vector storage (ChromaDB)
    \item \textbf{Extraction Layer:} Added context and preference extraction
    \item \textbf{Personalization Layer:} Added user-specific adaptation
    \item \textbf{Tool Enhancement:} Dynamic tool loading based on user context
\end{enumerate}

\subsection{Value Demonstration}

The context/memory system provides clear value:

\begin{enumerate}
    \item \textbf{Improved User Experience:}
    \begin{itemize}
        \item Users don't need to repeat information
        \item Responses are personalized and contextually relevant
        \item Long conversations maintain coherence
    \end{itemize}
    
    \item \textbf{Better Task Completion:}
    \begin{itemize}
        \item Higher accuracy on multi-turn tasks
        \item Successful tool selection based on context
        \item Complex planning tasks completed successfully
    \end{itemize}
    
    \item \textbf{Scalability:}
    \begin{itemize}
        \item Vector storage enables efficient semantic search
        \item User-specific namespaces prevent data leakage
        \item Persistent storage survives server restarts
    \end{itemize}
\end{enumerate}

\subsection{Statistical Summary}

Table~\ref{tab:summary} presents the comprehensive performance summary.

\begin{table*}[t]
\centering
\caption{Phase 2 Performance Summary}
\label{tab:summary}
\begin{tabular}{lccccc}
\toprule
Benchmark & Cases & Answer Accuracy & Response Quality & Tool Call Efficiency & Avg Latency \\
\midrule
Short & 6 & 100\% & 4.8/5.0 & 1.00 & 14.0s \\
Medium & 4 & 100\% & 4.775/5.0 & 1.00 & 12.0s \\
Long & 4 & 100\% & 4.65/5.0 & 0.714 & 17.0s \\
LoCoMo & 1 & 85.20\% & 4.29/5.0 & N/A & N/A \\
\bottomrule
\end{tabular}
\end{table*}

\textbf{Note:} LoCoMo benchmark answers do not require tool calls, and due to token limits we do not test its latency.

\section{Novelty and Technical Depth}

\subsection{Novel Contributions}

\subsubsection{Hybrid Context Extraction System}

\textbf{Novelty:} The system combines \textbf{pattern-based extraction} with \textbf{LLM-based structured extraction} for comprehensive preference detection.

\begin{itemize}
    \item \textbf{Pattern-Based} (\texttt{ContextExtractor}): Fast, rule-based extraction for common preference indicators
    \item \textbf{LLM-Based} (\texttt{extract\_preferences}): Deep semantic analysis using Claude's structured output for nuanced preferences
\end{itemize}

This hybrid approach balances speed and accuracy, allowing real-time extraction while maintaining high-quality preference detection.

\subsubsection{User-Specific Memory Namespaces with LangMem Integration}

\textbf{Novelty:} Integration of LangMem with ChromaDB backend, providing:
\begin{itemize}
    \item \textbf{Dynamic tool generation} per user
    \item \textbf{Namespace isolation} for multi-user scenarios
    \item \textbf{Unified API} for both agent tools and REST endpoints
\end{itemize}

The system creates user-specific memory tools (\texttt{search\_memory}, \texttt{store\_memory}) dynamically, ensuring complete isolation while sharing the same storage backend.

\subsubsection{Preference-Aware Tool Selection}

\textbf{Novelty:} Tool selection is influenced by extracted preferences:
\begin{itemize}
    \item Language preferences $\rightarrow$ Translator tool prioritization
    \item Communication style $\rightarrow$ Response format adaptation
    \item Domain interests $\rightarrow$ Relevant tool suggestions
\end{itemize}

This goes beyond simple tool availability to context-aware tool optimization.

\subsection{Technical Depth}

\subsubsection{Vector Storage Architecture}

\paragraph{Implementation Depth}
\begin{itemize}
    \item \textbf{ChromaDB Integration:} Full implementation of vector storage interface
    \item \textbf{Embedding Pipeline:} OpenAI embeddings with batch processing
    \item \textbf{Metadata Management:} Complex metadata flattening for ChromaDB compatibility
    \item \textbf{Query Optimization:} Efficient similarity search with configurable thresholds
\end{itemize}

\paragraph{Technical Challenges Solved}
\begin{itemize}
    \item Metadata flattening for ChromaDB's flat metadata structure
    \item User isolation using metadata filters
    \item Batch embedding generation for performance
    \item Error handling and connection management
\end{itemize}

\subsubsection{Asynchronous Architecture}

\paragraph{Implementation Depth}
\begin{itemize}
    \item Full async/await support throughout the stack
    \item Non-blocking memory operations
    \item Concurrent tool execution where possible
    \item Proper resource cleanup and error handling
\end{itemize}

\paragraph{Code Quality}
\begin{itemize}
    \item Type hints throughout
    \item Comprehensive error handling
    \item Structured logging with request tracking
    \item Modular design with clear interfaces
\end{itemize}

\subsubsection{Memory Management System}

\paragraph{Implementation Depth}
\begin{itemize}
    \item \textbf{Short-term memory:} In-memory conversation history
    \item \textbf{Long-term memory:} Persistent vector storage
    \item \textbf{Preference storage:} Structured preference extraction and storage
    \item \textbf{Session management:} Session-based document organization
\end{itemize}

\paragraph{Memory Lifecycle}
\begin{enumerate}
    \item Messages added to short-term memory
    \item Preferences extracted periodically
    \item Session summaries generated
    \item Long-term storage in ChromaDB with embeddings
    \item Semantic search for context retrieval
\end{enumerate}

\section{Conclusion}

The enhanced agent with context/memory implementation demonstrates significant improvements over the baseline:

\begin{enumerate}
    \item \textbf{High Accuracy:} 100\% on all benchmarks (short, medium, and long), representing improvements of +5.7\%, +33.3\%, and +78.6\% over Phase 1 baseline
    
    \item \textbf{High Response Quality:} Excellent quality ratings (4.65-4.8/5.0) across short, medium, and long benchmarks, representing improvements of +7.9\%, +24.0\%, and +114.3\% over Phase 1 baseline. LoCoMo benchmark achieved 4.29/5.0 quality rating.
    
    \item \textbf{Effective Personalization:} Successful preference extraction and application in 50\% of medium benchmark cases (2/4 cases)
    
    \item \textbf{Cross-Session Memory:} Functional long-term memory with 85.20\% overall accuracy (85-91\% in Categories 2-5)
    
    \item \textbf{Robust Architecture:} Modular design with proper error handling and resource management
    
    \item \textbf{Significant Improvements over Phase 1:} New capabilities in memory, personalization, and cross-session recall
\end{enumerate}

\paragraph{Key Achievements}
\begin{itemize}
    \item Comprehensive context extraction (pattern + LLM-based)
    \item Persistent vector storage with ChromaDB
    \item User-specific memory namespaces
    \item Preference-aware tool selection
    \item Cross-session memory retrieval
\end{itemize}

\paragraph{Future Work}
\begin{itemize}
    \item Category 1 (identity/personal) accuracy improvement
    \item Latency optimization through parallelization
    \item Advanced preference merging with temporal decay
    \item Context compression for very long conversations
\end{itemize}

The system provides a solid foundation for production deployment with measurable improvements in user experience, task completion, and personalization.

\appendix

\section{Experimental Results Summary}

\subsection{Short Benchmark Detailed Results}

Table~\ref{tab:short-detailed} presents the detailed results for each short benchmark test case.

\begin{table}[h]
\centering
\caption{Short Benchmark Detailed Results}
\label{tab:short-detailed}
\begin{tabular}{llcc}
\toprule
Test ID & Tools Used & Correct & Latency (s) \\
\midrule
test\_001 & calculator & \checkmark & 12.2 \\
test\_002 & get\_weather & \checkmark & 15.9 \\
test\_003 & translator & \checkmark & 13.7 \\
test\_004 & file\_system\_search & \checkmark & 18.7 \\
test\_005 & web\_reader & \checkmark & 10.4 \\
test\_006 & weather, translator & \checkmark & 17.1 \\
\bottomrule
\end{tabular}
\end{table}

\subsection{Medium Benchmark Detailed Results}

Table~\ref{tab:medium-detailed} shows the medium benchmark detailed results.

\begin{table}[h]
\centering
\caption{Medium Benchmark Detailed Results}
\label{tab:medium-detailed}
\begin{tabular}{lccc}
\toprule
Test ID & Turns & Correct & Preferences \\
\midrule
test\_006 & 4 & 4/4 & None \\
test\_007 & 3 & 3/3 & living\_place, hobbies \\
test\_008 & 2 & 2/2 & None \\
test\_009 & 3 & 3/3 & translation\_lang \\
\bottomrule
\end{tabular}
\end{table}

\subsection{Long Benchmark Detailed Results}

Table~\ref{tab:long-detailed} presents the long benchmark detailed results.

\begin{table}[h]
\centering
\caption{Long Benchmark Detailed Results}
\label{tab:long-detailed}
\begin{tabular}{lccc}
\toprule
Test ID & Turns & Correct & Memory Used \\
\midrule
test\_030 & 11 & 11/11 & Context retention \\
test\_031 & 4 & 4/4 & store\_memory \\
test\_032 & 11 & 11/11 & Preferences \\
test\_033 & 9 & 9/9 & Context retention \\
\bottomrule
\end{tabular}
\end{table}

\subsection{LoCoMo Memory Benchmark Results}

\paragraph{Storage Statistics}
\begin{itemize}
    \item Sessions: 19
    \item Total turns: 419
    \item Chunks stored: 144
    \item Characters stored: 67,162
\end{itemize}

\paragraph{QA Performance}
\begin{itemize}
    \item Total questions: 196
    \item Correct answers: 167
    \item Overall accuracy: 85.20\%
    \item Category 1 accuracy: 71.0\%
    \item Category 2-5 accuracy: 85-91\%
\end{itemize}

\end{document}

